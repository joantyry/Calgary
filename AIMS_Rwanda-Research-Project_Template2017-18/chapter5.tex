\chapter{Conclusion and Recommendations}
Security and privacy are the two most valued priorities among the many things we regard as
 important. When selecting a house location, we often put security as the major factor of consideration. You do not want someone to intrude your privacy or walk away with your hard earned cash or favorite items. Before telling our friends and family members something about ourselves we always consider whether they are in a position to keep it to themselves or not. We apply for jobs where we expose a lot of our details. We remain hopeful that none of this information will fall into the wrong hands. For instance, most ladies want to hide their age. After providing their information, they trust that the institution will handle it in a very secure and private manner.

There is a time people did not believe that we would actually have computers. Look at the way technology has change the world! Personal computers could no longer handle our problems. We needed to replace them with better machines which was expensive for large organizations. We therefore required to outsource resources. Cloud computing came along and we enjoyed it for a while until the issue of security popped up. There have been numerous trials to solve this issue like having several authentications before being granted access. The most interesting solution so far is Homomorphic Encryption. One can store, process and transmit encrypted data while still maintaining its confidentiality, privacy and security.

In this essay, a solution to the security issue while outsourcing and requiring services of untrusted server has been presented. Homomorphic Encryption is the process where by, computations are performed on encrypted data. The different solutions tried before Homomorphic Encryption such as multiple authentications, have their own limitations. All these solutions have loopholes like bypassing passwords which put security of the information at a risk. Homomorphic Encryption has proved to be a solution with close to no loopholes, unless someone steals your private key. A private key as we can recall is the key used for decryption. Data is encrypted in storage, transit and during computations.

Two examples, a school and a hospital are given. These are places where Quantum Homomorphic Encryption (QHE) could be useful. QHE has several advantages over Classical Homomorphic Encryption (CHE) like better computational power, speed and efficiency. Quantum states cannot also be duplicated due to their non-cloning property. Cloning involves measurement and measurement changes the state as illustrated in the essay. These advantages of QHE over CHE make QHE more attractive. A comparison between Blind Quantum Computing (BQC) and QHE is clearly outlined. The main disadvantage of BQC over QHE is cost of implementation. We want more clients to be involved thus QHE is the most attractive solution to our problem.

Clearly QHE is possible. We have seen it can be implemented using a classical client and a quantum server. The problem is, there are very few quantum computers in the world hence testing and implementation is difficult. The number of people with quantum knowledge are also few in the world. This is a major setback in this area of study. This work is mainly theoretical. We need more people to enroll in quantum courses and help develop more quantum computers. With an increase in the number of quantum computers, research and experiments will be easy to perform. Different people come up with different ideas, thus, an increase to the number of people with quantum knowledge will lead to great improvements and new innovations. This will make cloud computing environments more reliable and popular among institutions dealing with very delicate data. Major improvements can also be made in the examples given and spread in other areas not mentioned. 









